% Capstone Report Template - Basic Structure
% This is a minimal template to get you started with the capstone_report.cls
% For a complete feature demonstration, see capstonereporttest.tex

\documentclass[titlecase]{capstone_report}

\begin{document}

% Progress report metadata
% \progressreport{report#}{name}{team}{team#}{start-date}{end-date}[course]
\progressreport{01}{Your Name}{Team/Project Name}{12}{Sep 1, 2025}{Sep 14, 2025}[Course Name: Capstone Design]
\makeheader

% Standard report sections - fill in as needed

\reportIntroduction
Provide a brief introduction to your project and this report period.

\reportProjectDescription
Describe your capstone project, its goals, and scope.

\reportProgressSummary
Summarize the overall progress made during this reporting period.

\reportMyPlan
Describe your specific plan and responsibilities for this period.

\reportWorkAccomplished
Detail the work you completed during this reporting period.

\reportProblems
Discuss any problems, challenges, or roadblocks encountered.

\reportPlans
Outline your plans for the next reporting period.

\reportReqChanges
Note any changes to project requirements or scope.

\reportAssessment
Provide your assessment of project progress and timeline.

\reportInstructorIssues
List any questions or issues that need instructor attention.

% Optional: Include code examples using the provided environments
% \begin{reportpython}[caption=Example Python Code]
% import numpy as np
% data = np.array([1, 2, 3, 4, 5])
% print(f"Mean: {np.mean(data)}")
% \end{reportpython}

% Optional: Include figures
% \begin{figure}[htb]
%     \centering
%     \includegraphics[width=0.6\textwidth]{path/to/figure.pdf}
%     \caption{Example figure}
%     \label{fig:example}
% \end{figure}
% 
% Note: Use \includegraphics for including external figures and diagrams

% Optional: Add appendices if needed
% \reportappendix{Code Listings}
% Additional code, data, or detailed information can go here.

\end{document}