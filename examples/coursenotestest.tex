\documentclass{coursenotes}

\begin{document}

% ============================================================================
% COURSE HEADER
% ============================================================================
\courseheader{EE 571}{Control Systems}{Fall 2025}{Student Name}

% ============================================================================
% FRONT MATTER - Table of Contents, List of Figures, List of Tables
% ============================================================================
\makealllists

% ============================================================================
% CHAPTER 1: TIME DOMAIN ANALYSIS
% ============================================================================
\chapter{Time Domain Analysis} \label{chap:time}

\section{First-Order Systems} \label{sec:firstorder}

A first-order system has the general transfer function:

\begin{hwmath}
G(s) \eq \frac{K}{\tau s + 1}
\end{hwmath}

where $K$ is the DC gain and $\tau$ is the time constant. The step response of a first-order system is characterized by:

\begin{itemize}
\item Time constant $\tau$ - time to reach 63.2\% of final value
\item Settling time $t_s \approx 4\tau$ - time to reach within 2\% of final value
\item Rise time $t_r \approx 2.2\tau$ - time from 10\% to 90\%
\end{itemize}

\note[IMPORTANT: First-order systems are always stable if $\tau > 0$]

\begin{example}{Example 1-1: RC Circuit Analysis}
Consider an RC circuit with $R = 10k\Omega$ and $C = 10\mu F$. Find the time constant and settling time.

\textbf{Solution:}

The time constant is:
\begin{hwmath}
\tau \eq RC \\
\eq (10 \times 10^3)(10 \times 10^{-6}) \\
\eq 0.1 \text{ seconds}
\end{hwmath}

The settling time is:
\begin{hwmath}
t_s \eq 4\tau \\
\eq 4(0.1) \\
\eq 0.4 \text{ seconds}
\end{hwmath}
\end{example}

Reference: \lectureref{3a}, \textbookref{§2.3}

\subsection{MATLAB Simulation}

We can simulate the step response using MATLAB:

\begin{hwmatlab}[caption=First-Order System Simulation]
% Define system parameters
K = 5;
tau = 0.1;

% Create transfer function
s = tf('s');
G = K/(tau*s + 1);

% Plot step response
step(G);
grid on;
title('Step Response of First-Order System');
\end{hwmatlab}

\subsection{Python Implementation}

Alternatively, use Python with the control package:

\begin{hwpython}[caption=Python Control System Analysis]
import control as ct
import numpy as np
import matplotlib.pyplot as plt

# System parameters
K = 5
tau = 0.1

# Create transfer function
num = [K]
den = [tau, 1]
G = ct.tf(num, den)

# Step response
t, y = ct.step_response(G)
plt.plot(t, y)
plt.grid(True)
plt.xlabel('Time (s)')
plt.ylabel('Output')
plt.title('Step Response')
plt.show()
\end{hwpython}

\section{Second-Order Systems} \label{sec:secondorder}

Second-order systems have the standard form:

\begin{hwmathnumbered}
G(s) \eq \frac{\omega_n^2}{s^2 + 2\zeta\omega_n s + \omega_n^2} \label{eq:secondorder}
\end{hwmathnumbered}

where $\omega_n$ is the natural frequency and $\zeta$ is the damping ratio.

The system behavior depends on the damping ratio:
\begin{itemize}
\item $\zeta \gt 1$: Overdamped
\item $\zeta \eq 1$: Critically damped
\item $0 \lt \zeta \lt 1$: Underdamped
\item $\zeta \eq 0$: Undamped
\end{itemize}

As shown in \secref{sec:firstorder}, first-order systems are simpler, but second-order systems (Equation~\ref{eq:secondorder}) are more common in practice.

Reference: \lectureref{4b}, \textbookref{pp. 87-92}

% ============================================================================
% PROBLEM SET 1
% ============================================================================

\problem{First-Order System Analysis} \label{prob:first}

Consider a first-order system with transfer function $G(s) = \frac{5}{0.2s + 1}$.

\subproblem{Calculate the time constant.}
\begin{solution}
Comparing with the standard form $G(s) = \frac{K}{\tau s + 1}$, we have $\tau = 0.2$ seconds.
\end{solution}

\subproblem{Find the DC gain.}
\begin{solution}
The DC gain is $K = 5$, found by evaluating $G(0) = \frac{5}{1} = 5$.
\end{solution}

\subproblem{Determine the settling time.}
\begin{solution}
The settling time is:
\begin{hwmath}
t_s \eq 4\tau \\
\eq 4(0.2) \\
\eq 0.8 \text{ seconds}
\end{hwmath}
\end{solution}

\problem{Second-Order System Parameters}

A second-order system has $\omega_n = 10$ rad/s and $\zeta = 0.5$.

\subproblem{Write the transfer function.}
\begin{solution}
Using the standard form:
\begin{hwmath}
G(s) \eq \frac{\omega_n^2}{s^2 + 2\zeta\omega_n s + \omega_n^2} \\
\eq \frac{100}{s^2 + 10s + 100}
\end{hwmath}
\end{solution}

\subproblem{Find the damped natural frequency.}
\begin{solution}
The damped natural frequency is:
\begin{hwmath}
\omega_d \eq \omega_n\sqrt{1 - \zeta^2} \\
\eq 10\sqrt{1 - 0.5^2} \\
\eq 10\sqrt{0.75} \\
\approxx 8.66 \text{ rad/s}
\end{hwmath}
\end{solution}

% Change sub-problem style to roman numerals for next problem
\setsubproblemstyle{roman}

\problem{System Stability Analysis}

\subproblem{What condition ensures stability?}
\begin{solution}
For stability, all poles must have negative real parts (located in the left half-plane).
\end{solution}

\subproblem{How does damping ratio affect stability?}
\begin{solution}
Systems with $\zeta > 0$ are stable. As $\zeta$ increases from 0 to 1, the system transitions from oscillatory to smooth response.
\end{solution}

% ============================================================================
% CHAPTER 2: FREQUENCY DOMAIN ANALYSIS
% ============================================================================
\chapter{Frequency Domain Analysis} \label{chap:frequency}

\section{Bode Plots} \label{sec:bode}

Bode plots are graphical representations of system frequency response, consisting of:
\begin{itemize}
\item Magnitude plot: $20\log_{10}|G(j\omega)|$ vs $\omega$
\item Phase plot: $\angle G(j\omega)$ vs $\omega$
\end{itemize}

Reference: \lectureref{7}, \textbookref{Chapter 4}

\subsection{Stability Margins}

From Bode plots, we can determine:

\begin{hwmath}
\text{Gain Margin (GM)} \eq \frac{1}{|G(j\omega_{pc})|}
\end{hwmath}

where $\omega_{pc}$ is the phase crossover frequency ($\angle G(j\omega_{pc}) = -180°$).

\begin{hwmath}
\text{Phase Margin (PM)} \eq 180° + \angle G(j\omega_{gc})
\end{hwmath}

where $\omega_{gc}$ is the gain crossover frequency ($|G(j\omega_{gc})| = 1$).

For stable systems:
\begin{hwmath}
\text{GM} \ggt 1 \text{ (or GM} \ggt 6 \text{ dB)} \\
\text{PM} \ggt 30°
\end{hwmath}

\section{Block Diagrams} \label{sec:blockdiagrams}

Consider a unity feedback control system:

\begin{hwblocks}[scale=0.75]
\bXInput[r(s)]{A}
\bXComp{B}{A}
\bXBlocL[2]{C}{$G_c(s)$}{B}
\bXBlocL[2]{D}{$G_p(s)$}{C}
\bXOutput[3]{E}{D}
\bXLink[$r$]{A}{B}
\bXLink[$e$]{B}{C}
\bXLink[$u$]{C}{D}
\bXLink[$y$]{D}{E}
\bXReturn{D-E}{B}{}
\end{hwblocks}

The closed-loop transfer function is:

\begin{hwmath}
T(s) \eq \frac{G_c(s)G_p(s)}{1 + G_c(s)G_p(s)}
\end{hwmath}

As discussed in \chapref{chap:time}, time domain characteristics can be related to frequency domain properties. The relationship between damping ratio and phase margin is approximately:

\begin{hwmath}
\zeta \approxx \frac{\text{PM}}{100}
\end{hwmath}

for phase margins between 0° and 60°.

Reference: \lectureref{8a}, \textbookref{§4.5}

% ============================================================================
% PROBLEM SET 2
% ============================================================================

% Reset to letter style for new chapter
\setsubproblemstyle{alph}

\problem{Bode Plot Analysis} \label{prob:bode}

Given a system with transfer function $G(s) = \frac{100}{s(s+10)}$.

\subproblem{Find the gain crossover frequency.}
\begin{solution}
At the gain crossover frequency, $|G(j\omega_{gc})| = 1$. Solving:
\begin{hwmath}
\left|\frac{100}{j\omega_{gc}(j\omega_{gc}+10)}\right| \eq 1 \\
\frac{100}{\omega_{gc}\sqrt{\omega_{gc}^2 + 100}} \eq 1 \\
\omega_{gc} \approxx 3.01 \text{ rad/s}
\end{hwmath}
\end{solution}

\subproblem{Calculate the phase margin.}
\begin{solution}
The phase at $\omega_{gc}$ is:
\begin{hwmath}
\angle G(j\omega_{gc}) \eq -90° - \arctan\left(\frac{\omega_{gc}}{10}\right) \\
\approxx -90° - 16.7° \\
\eq -106.7°
\end{hwmath}

Therefore, the phase margin is:
\begin{hwmath}
\text{PM} \eq 180° + \angle G(j\omega_{gc}) \\
\eq 180° - 106.7° \\
\eq 73.3°
\end{hwmath}
\end{solution}

\subproblem{Is the system stable?}
\begin{solution}
Yes, the system is stable because:
\begin{itemize}
\item All poles are in the left half-plane
\item Phase margin is positive (73.3° $>$ 0°)
\item Phase margin exceeds the recommended minimum (73.3° $>$ 30°)
\end{itemize}
\end{solution}

\problem{PID Controller Design}

Design a PID controller for the plant $G_p(s) = \frac{10}{s(s+2)}$ to meet:
\begin{itemize}
\item Settling time $t_s < 2$ seconds
\item Zero steady-state error for step input
\item Phase margin $>$ 45°
\end{itemize}

\subproblem{Why is the integral term necessary?}
\begin{solution}
The integral term is necessary to eliminate steady-state error for step inputs. Since the plant already has one integrator ($1/s$ term), we need to maintain at least one integrator in the overall loop to ensure zero steady-state error.
\end{solution}

\subproblem{What is the role of the derivative term?}
\begin{solution}
The derivative term provides phase lead, which increases the phase margin and improves transient response. It adds damping to the system, reducing overshoot and oscillations.
\end{solution}

\subproblem{Propose initial PID gains.}
\begin{solution}
Starting with Ziegler-Nichols tuning or trial-and-error:
\begin{itemize}
\item $K_p = 5$ (proportional gain)
\item $K_i = 10$ (integral gain) 
\item $K_d = 1$ (derivative gain)
\end{itemize}

The PID controller is:
\begin{hwmath}
G_c(s) \eq K_p + \frac{K_i}{s} + K_d s \\
\eq 5 + \frac{10}{s} + s \\
\eq \frac{s^2 + 5s + 10}{s}
\end{hwmath}

These gains should be refined using simulation and frequency response analysis.
\end{solution}

\problem{Root Locus Application}

This problem demonstrates how to use root locus for controller design. Refer to Problem~\ref{prob:first} in \chapref{chap:time} for background on system parameters.

\subproblem{Sketch the root locus for $G(s) = \frac{K}{s(s+2)}$.}
\begin{solution}
The root locus has:
\begin{itemize}
\item Two poles at $s = 0$ and $s = -2$
\item No finite zeros
\item Two branches to infinity at angles $\pm 90°$
\item Breakaway point at $s = -1$
\end{itemize}

For stability, we require $K > 0$. The system is stable for all positive $K$ values.
\end{solution}

\subproblem{What gain gives $\zeta = 0.707$?}
\begin{solution}
For $\zeta = 0.707$, the closed-loop poles should be at $45°$ angles from the real axis. From the root locus, this occurs at approximately $K = 2$.

The closed-loop poles are at:
\begin{hwmath}
s \eq -1 \pm j
\end{hwmath}

which corresponds to $\omega_n = \sqrt{2}$ rad/s and $\zeta = 0.707$.
\end{solution}

% ============================================================================
% SUMMARY AND CROSS-REFERENCES
% ============================================================================

\section{Summary}

This chapter covered frequency domain analysis methods including:
\begin{itemize}
\item Bode plots (\secref{sec:bode})
\item Stability margins and their interpretation
\item Block diagram analysis (\secref{sec:blockdiagrams})
\item Relationship to time domain properties from \chapref{chap:time}
\end{itemize}

Key takeaways:
\begin{enumerate}
\item Phase margin $\ggt$ 45° ensures good transient response
\item Gain margin $\ggt$ 6 dB provides robustness
\item Frequency domain and time domain are linked through system parameters
\end{enumerate}

\note[Remember: Always verify stability margins in frequency domain analysis]

Reference: \lectureref{9}, \textbookref{Chapter 4 Summary}

\end{document}
