\documentclass{../designreport}

% Required packages for bibliography
\addbibresource{references.bib}

% Document information
\title{Design Report Test Document}
\author{Test Author}
\documentname{Design Report}
\teamname{Test Team}
\teamlogo{../figures/logo.png}  
\university{Your University Name}
\department{Department Name}
\course{Test Course}
\teammembers{
    \teammember{person 1}{person1@uni.edu}
    \teammember{person 2}{person2@uni.edu}
    \teammember{person 3}{person3@uni.edu}
    \teammember{person 4}{person4@uni.edu}
}

% Class advisors
\classadvisors{%
    \advisor{person 5}{person5@uni.edu}
    \advisor{person 6}{person6@uni.edu}
    \advisor{person 7}{person7@uni.edu}
    \advisor{person 8}{person8@uni.edu}
    \advisor{person 9}{person9@uni.edu}
    \advisor{person 10}{person10@uni.edu}

}

\sponsors{%
    \sponsor{person 11}{person11@uni.edu}
    \sponsor{person 12}{person12@uni.edu}
}
\reportdate{\today}

\begin{document}

\maketitle

\tableofcontents
\newpage

\section{Introduction}

This is a test document to verify that the \texttt{designreport.cls} class file works correctly after removing the impact statement environment.

\section{Test Section}

This section contains some basic content to test the formatting:

\subsection{Subsection Test}

This is a subsection with some text content.

\subsubsection{Subsubsection Test}

This is a subsubsection to test the formatting hierarchy.

\section{Engineering Requirements}

The following table demonstrates the longtable functionality for engineering requirements that can span multiple pages:

\begin{longtable}{|p{1cm}|p{4cm}|p{1cm}|p{4cm}|p{4cm}|}
\caption{Engineering Requirements Table} \label{tab:engineering-requirements} \\
\hline
\textbf{ER No.} & \textbf{Engineering Requirement} & \textbf{MR} & \textbf{Justification} & \textbf{Verification} \\
\hline
\endfirsthead
\hline
\textbf{ER No.} & \textbf{Engineering Requirement} & \textbf{MR} & \textbf{Justification} & \textbf{Verification} \\
\hline
\endhead
\hline
\endfoot
\hline
\endlastfoot
1 & System shall control primary functionality & 1 & Core objective of the project & Test each control mechanism individually \\
\hline
2 & Device shall fit within specified dimensions & 2 & Physical constraints of the application & Verify using dimensional measurements \\
\hline
3 & System shall have independent power supply & 3 & Operational independence requirement & Confirm electrical isolation using multimeter \\
\hline
4 & Device shall sense motion and orientation & 1,5 & Required for state monitoring & Perform sensor accuracy and repeatability tests \\
\hline
5 & System shall operate for required duration & 1,3 & Must function throughout operation period & Test under simulated conditions \\
\hline
6 & Device shall use standard components & 3 & Standardization and availability & Verify component specifications \\
\hline
7 & System shall not cause interference & 4 & Compatibility with other systems & Perform electromagnetic compatibility testing \\
\hline
8 & Device shall withstand operational forces & 5 & Structural integrity under normal loads & Conduct force testing per specifications \\
\hline
9 & System shall tolerate vibrations & 5 & Environmental durability requirement & Perform vibration testing at specified levels \\
\hline
10 & Device shall handle impact loads & 5 & Shock resistance capability & Execute impact testing per mission profile \\
\hline
11 & System shall operate in thermal range & 5 & Temperature tolerance requirement & Test across expected temperature range \\
\hline
12 & Development cost shall stay within budget & 7 & Financial constraint & Track expenses against allocated budget \\
\hline
13 & Design shall comply with regulations & 2 & Legal and regulatory compliance & Review against applicable standards \\
\hline
14 & System shall provide real-time response & 6 & Performance timing requirement & Verify response time under test conditions \\
\hline
15 & Device shall not interfere with operations & 6 & Non-interference requirement & Test activation scenarios and verify non-operation \\
\hline
\end{longtable}

\section{Conclusion}

The design report class file appears to be working correctly with longtable support.

\end{document}