\documentclass{coursenotes}

\begin{document}

% ============================================================================
% COURSE HEADER
% Replace with your course information
% ============================================================================
\courseheader{COURSE CODE}{Course Name}{Semester Year}{Your Name}

% ============================================================================
% GENERATE FRONT MATTER
% Automatically creates Table of Contents, List of Figures, List of Tables
% Comment out any you don't need
% ============================================================================
\makealllists

% Or generate individually:
% \maketoc  % Table of Contents only
% \makelof  % List of Figures only
% \makelot  % List of Tables only

% ============================================================================
% CHAPTER 1
% Use \label{chap:uniquename} to reference this chapter later
% ============================================================================
\chapter{Chapter Title Here} \label{chap:chapter1}

% Add lecture and textbook references at the start of each chapter/section
% Reference: \lectureref{1}, \textbookref{Chapter 1}

\section{Section Title} \label{sec:section1}

Write your lecture notes content here. You can include:
\begin{itemize}
\item Bullet points for key concepts
\item Mathematical derivations
\item Code examples
\item Block diagrams
\item Figures and graphs
\end{itemize}

% Reference: \lectureref{1a}, \textbookref{§1.2}

\subsection{Subsection Title}

More detailed content...

% ----------------------------------------------------------------------------
% MATH ENVIRONMENT EXAMPLE
% ----------------------------------------------------------------------------

Use the hwmath environment for unnumbered aligned equations:

\begin{hwmath}
y(t) \eq Ax(t) + Bu(t) \\
\eq \begin{bmatrix} a_{11} & a_{12} \\ a_{21} & a_{22} \end{bmatrix} x(t) + \begin{bmatrix} b_1 \\ b_2 \end{bmatrix} u(t)
\end{hwmath}

For numbered equations (can be labeled and referenced):

\begin{hwmathnumbered}
G(s) \eq \frac{Y(s)}{U(s)} \label{eq:transfer}
\end{hwmathnumbered}

Later reference it as: See Equation~\ref{eq:transfer} for the transfer function.

% Math shortcuts available: \eq (=), \gt (>), \lt (<), \ggt (>>), \llt (<<),
% \geqq (≥), \leqq (≤), \neqq (≠), \approxx (≈)

% ----------------------------------------------------------------------------
% NOTE COMMAND EXAMPLE
% ----------------------------------------------------------------------------

\note[IMPORTANT: Always check stability conditions]

% Or just: \note

% ----------------------------------------------------------------------------
% EXAMPLE BOX
% ----------------------------------------------------------------------------

\begin{example}{Example 1-1: Problem Title}
State the problem here...

\textbf{Solution:}

Work through the solution step by step...

\begin{hwmath}
\text{Result} \eq \text{Final Answer}
\end{hwmath}
\end{example}

% ----------------------------------------------------------------------------
% CODE ENVIRONMENT EXAMPLES
% ----------------------------------------------------------------------------

\subsection{MATLAB Code}

\begin{hwmatlab}[caption=Description of Code]
% Your MATLAB code here
s = tf('s');
G = 1/(s+1);
step(G);
\end{hwmatlab}

\subsection{Python Code}

\begin{hwpython}[caption=Python Implementation]
# Your Python code here
import numpy as np
import matplotlib.pyplot as plt

x = np.linspace(0, 10, 100)
y = np.sin(x)
plt.plot(x, y)
plt.show()
\end{hwpython}

\subsection{Terminal Output}

\begin{hwterminal}[caption=Command Line Example]
$ pip install package-name
Collecting package-name
Installing...
Successfully installed package-name
\end{hwterminal}

% ----------------------------------------------------------------------------
% GRAPHICS EXAMPLES
% ----------------------------------------------------------------------------

\section{Figures and Graphics}

% Single figure with title and scale
% \hwgraphic{path/to/image.pdf}[Figure Title][0.7]

% Side-by-side figures
% \hwdualfigure{image1.pdf}{image2.pdf}[0.45\textwidth][0.45\textwidth]{Caption 1}{Caption 2}

% ----------------------------------------------------------------------------
% BLOCK DIAGRAM EXAMPLE
% ----------------------------------------------------------------------------

\section{Block Diagrams}

\begin{hwblocks}[scale=0.8]
% Simple feedback system example
\bXInput[r(s)]{A}
\bXComp{B}{A}
\bXBlocL[2]{C}{$G_c(s)$}{B}
\bXBlocL[2]{D}{$G_p(s)$}{C}
\bXOutput[3]{E}{D}
\bXLink[$r$]{A}{B}
\bXLink[$e$]{B}{C}
\bXLink[$u$]{C}{D}
\bXLink[$y$]{D}{E}
\bXReturn{D-E}{B}{}
\end{hwblocks}

% See coursenotes-guide.md or homework-guide.md for complete blox documentation

% ============================================================================
% PROBLEM SET FOR CHAPTER 1
% Add problems at the end of each chapter
% ============================================================================

\problem{Problem Title Here} \label{prob:problem1}

State the problem...

\subproblem{Part (a) question here}
\begin{solution}
Write your solution to part (a) here. It will display inline after the problem statement.

You can include math:
\begin{hwmath}
\text{Answer} \eq \text{Value}
\end{hwmath}
\end{solution}

\subproblem{Part (b) question here}
\begin{solution}
Write your solution to part (b) here.
\end{solution}

\subproblem{Part (c) question here}
\begin{solution}
Write your solution to part (c) here.
\end{solution}

% ----------------------------------------------------------------------------
% ADDITIONAL PROBLEMS
% ----------------------------------------------------------------------------

\problem{Another Problem Title}

Another problem statement...

\subproblem{First part}
\begin{solution}
Solution here...
\end{solution}

\subproblem{Second part}
\begin{solution}
Solution here...
\end{solution}

% To change sub-problem style (default is letters: a, b, c)
% \setsubproblemstyle{roman}  % for (i), (ii), (iii)
% \setsubproblemstyle{arabic} % for (1), (2), (3)
% \setsubproblemstyle{alph}   % back to (a), (b), (c)

% ============================================================================
% CHAPTER 2
% ============================================================================
\chapter{Second Chapter Title} \label{chap:chapter2}

% Reference: \lectureref{5}, \textbookref{Chapter 2}

\section{Section in Chapter 2}

Content for second chapter...

You can reference earlier chapters: As discussed in \chapref{chap:chapter1}, ...

You can reference earlier sections: See \secref{sec:section1} for background.

You can reference earlier problems: As solved in Problem~\ref{prob:problem1}, ...

% ----------------------------------------------------------------------------
% MORE PROBLEMS
% Problems automatically number as 2.1, 2.2, etc. in Chapter 2
% ----------------------------------------------------------------------------

\problem{Chapter 2 Problem}

\subproblem{Question}
\begin{solution}
Answer...
\end{solution}

% ============================================================================
% ADD MORE CHAPTERS AS NEEDED
% ============================================================================

% \chapter{Third Chapter Title} \label{chap:chapter3}
% ... content ...

% ============================================================================
% TIPS FOR USING THIS TEMPLATE
% ============================================================================
% 
% 1. Replace all placeholder text with your actual content
% 2. Add \label{} to chapters, sections, equations, problems you want to reference
% 3. Use descriptive label names: chap:name, sec:name, eq:name, prob:name, fig:name
% 4. Add \lectureref{} and \textbookref{} to track source material
% 5. Compile with pdflatex at least twice for proper cross-references
% 6. Use \chapref{}, \secref{}, \ref{} for cross-references
% 7. Sub-problems default to (a), (b), (c) - use \setsubproblemstyle{} to change
% 8. Solutions display inline immediately after each sub-problem
% 9. See coursenotes-guide.md for complete documentation
% 10. See coursenotestest.tex for comprehensive examples
%
% ============================================================================

\end{document}
