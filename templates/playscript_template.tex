% Playscript Template - Basic Structure
% This is a minimal template to get you started with playscript.cls
% For a complete feature demonstration, see examples/playscripttest.tex

\documentclass{../playscript}

% Optional class options:
% \documentclass[font=sans]{playscript}      % Use sans-serif font
% \documentclass[a4paper]{playscript}        % Use A4 paper size
% \documentclass[draft]{playscript}          % Draft mode with overfull box markers

% ============================================================================
% TITLE PAGE INFORMATION
% ============================================================================

\title{Your Play Title}
\subtitle{(Optional Subtitle)}                    % Optional
\author{Your Name}
\translatedby{Translated by...}                   % Optional - for translations
\performancenote{Optional performance note}       % Optional - epigraph or note

\begin{document}

% ============================================================================
% FRONT MATTER
% ============================================================================

% Title page
\maketitle

% Character list
\begin{characters}
\character{FIRST CHARACTER}
\character{SECOND CHARACTER}
\character{THIRD CHARACTER}
% Add more characters as needed
\end{characters}

% ============================================================================
% ACT I
% ============================================================================

\act{I}

% Scene heading (optional - for scene location)
\scene{A room in the house}

% Scene description (optional - for detailed scene setting)
\begin{scenedesc}
The scene is set in a comfortable living room. A sofa sits center stage,
with a coffee table in front of it. Large windows at the back look out
onto a garden. The lighting is warm and inviting.
\end{scenedesc}

% ----------------------------------------------------------------------------
% BASIC DIALOGUE PATTERN
% ----------------------------------------------------------------------------

% Character name followed by dialogue
\speaker{FIRST CHARACTER}

\begin{dialogue}
This is how you write dialogue. The character's name appears centered
above the dialogue block, and the dialogue itself is left-justified
with appropriate margins for a traditional playscript appearance.
\end{dialogue}

% Character responding
\speaker{SECOND CHARACTER}

\begin{dialogue}
And this is how another character responds. Each new speaker gets
their name displayed before their dialogue.
\end{dialogue}

% ----------------------------------------------------------------------------
% DIALOGUE WITH PARENTHETICAL (CHARACTER DIRECTION)
% ----------------------------------------------------------------------------

\speaker{FIRST CHARACTER}
\paren{standing, excited}

\begin{dialogue}
Parentheticals are used for brief character actions or delivery notes.
They appear in italics with extra indentation.
\end{dialogue}

% ----------------------------------------------------------------------------
% LONGER PARENTHETICAL USING ENVIRONMENT
% ----------------------------------------------------------------------------

\begin{parenthetical}
For longer character directions that need more space, use the
parenthetical environment. This is useful for complex actions
or detailed delivery instructions.
\end{parenthetical}

\speaker{SECOND CHARACTER}

\begin{dialogue}
I understand completely.
\end{dialogue}

% ----------------------------------------------------------------------------
% STAGE DIRECTIONS
% ----------------------------------------------------------------------------

% Short stage direction (centered)
\stagedirection{THIRD CHARACTER enters from stage left}

% Longer stage direction using environment (left-justified block)
\begin{stage}
FIRST CHARACTER crosses to the window and looks out at the garden.
The lights dim slightly, creating a more intimate atmosphere.
A clock chimes in the distance.
\end{stage}

% ----------------------------------------------------------------------------
% THREE-WAY CONVERSATION
% ----------------------------------------------------------------------------

\speaker{THIRD CHARACTER}

\begin{dialogue}
I hope I'm not interrupting.
\end{dialogue}

\speaker{FIRST CHARACTER}
\paren{turning from window}

\begin{dialogue}
Not at all. Please, join us.
\end{dialogue}

\speaker{SECOND CHARACTER}

\begin{dialogue}
We were just discussing the matter.
\end{dialogue}

% ----------------------------------------------------------------------------
% LONGER MONOLOGUE
% ----------------------------------------------------------------------------

\speaker{THIRD CHARACTER}

\begin{dialogue}
I've been thinking about this for some time now, and I believe
I've come to understand what's really happening here. It's not
just about the surface issues--there's something deeper at work.
Something that connects us all in ways we haven't fully recognized.
\end{dialogue}

\vspace{0.5em}

\begin{dialogue}
And that's why I've come here today. To share what I've learned,
and to ask for your help in moving forward.
\end{dialogue}

% ----------------------------------------------------------------------------
% USING BEAT (PAUSE)
% ----------------------------------------------------------------------------

\speaker{FIRST CHARACTER}

\begin{dialogue}
I see.
\end{dialogue}

\beat

\begin{dialogue}
What exactly do you propose?
\end{dialogue}

% ----------------------------------------------------------------------------
% EMOTIONAL DIALOGUE WITH EMPHASIS
% ----------------------------------------------------------------------------

\speaker{THIRD CHARACTER}
\paren{earnestly}

\begin{dialogue}
We need to work together. Not as separate individuals, but as
a \emph{united} group. Only then can we hope to succeed.
\end{dialogue}

% ----------------------------------------------------------------------------
% COMPLEX STAGE DIRECTION
% ----------------------------------------------------------------------------

\begin{stage}
There is a long pause. The three characters look at one another,
each weighing the implications of what has been said. SECOND CHARACTER
rises and moves center stage, as if to speak, then hesitates.
Finally, they make their decision.
\end{stage}

\speaker{SECOND CHARACTER}

\begin{dialogue}
Very well. I'm with you.
\end{dialogue}

\speaker{FIRST CHARACTER}
\paren{relieved}

\begin{dialogue}
As am I. When do we begin?
\end{dialogue}

\speaker{THIRD CHARACTER}
\paren{smiling}

\begin{dialogue}
Now. We begin now.
\end{dialogue}

% End of act curtain
\curtain

% ============================================================================
% ACT II (if needed)
% ============================================================================

\act{II}

\scene{The same room, later that evening}

\begin{scenedesc}
The room is now lit only by lamplight. The atmosphere is more intimate
and shadowy. The characters have gathered around the coffee table.
\end{scenedesc}

\speaker{FIRST CHARACTER}

\begin{dialogue}
So we're agreed then?
\end{dialogue}

\speaker{SECOND CHARACTER}

\begin{dialogue}
We are.
\end{dialogue}

% Add more dialogue and scenes as needed...

% ----------------------------------------------------------------------------
% SCENE TRANSITION WITHIN ACT
% ----------------------------------------------------------------------------

\vspace{2em}

\scene{A different location, simultaneous}

\begin{scenedesc}
A stark contrast to the previous scene. This space is cold and
impersonal, suggesting danger or opposition to our protagonists.
\end{scenedesc}

% Continue with more dialogue...

% ----------------------------------------------------------------------------
% CONTINUING DIALOGUE INDICATOR
% ----------------------------------------------------------------------------
% Use when a character's dialogue continues from a previous page

\speaker{FIRST CHARACTER}
\continuing

\begin{dialogue}
As I was saying before we were interrupted...
\end{dialogue}

% End of play
\fadeout

\vspace{2em}

\begin{center}
\textsc{End of Play}
\end{center}

\end{document}

% ============================================================================
% QUICK REFERENCE
% ============================================================================
%
% STRUCTURE COMMANDS:
%   \act{I}                          - Act heading with page break
%   \scene{description}              - Scene location/description
%   \begin{scenedesc}...\end{...}    - Detailed scene description
%
% CHARACTER AND DIALOGUE:
%   \speaker{NAME}                   - Character name (centered, caps)
%   \character{NAME}                 - Alias for \speaker
%   \begin{dialogue}...\end{...}     - Dialogue block
%
% DIRECTIONS:
%   \paren{text}                     - Brief parenthetical direction
%   \begin{parenthetical}...\end{...}- Longer character direction
%   \stagedirection{text}            - Short stage direction
%   \begin{stage}...\end{...}        - Longer stage direction block
%
% SPECIAL COMMANDS:
%   \beat                            - Pause in dialogue
%   \continuing                      - Dialogue continues from prev page
%   \emph{text}                      - Emphasis within dialogue
%   \curtain                         - End of act curtain
%   \fadein                          - Fade in transition
%   \fadeout                         - Fade out transition
%
% COMPILATION:
%   pdflatex yourscript.tex
%   or
%   latexmk -pdf yourscript.tex
%
% ============================================================================
