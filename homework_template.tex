% Homework Template - Basic Structure
% This is a minimal template to get you started with the homework.cls
% For a complete feature demonstration, see homeworktest.tex

\documentclass{homework}

% Document metadata
\course{Course Name}
\title{Assignment Title}
\author{Your Name}
\email{your.email@domain.edu}
\duedate{Due Date}

\begin{document}

\makeheader

\problem{Problem Title}

Your solution goes here. You can use inline math like $x = 5$ or display math:
\begin{align}
    f(x) &= x^2 + 2x + 1 \\
    &= (x + 1)^2
\end{align}

\subproblem{Sub-problem A}

Sub-problem solution here.

\subproblem{Sub-problem B}

Another sub-problem solution.

\problem{Code Example}

You can include code using the provided environments:

\begin{matlab}
% MATLAB code example
x = linspace(0, 2*pi, 100);
y = sin(x);
plot(x, y);
\end{matlab}

\begin{python}
# Python code example
import numpy as np
x = np.linspace(0, 2*np.pi, 100)
y = np.sin(x)
\end{python}

\begin{terminal}
$ python script.py
Output appears here
\end{terminal}

\problem{Including Graphics}

You can include figures and reference them:

% \begin{figure}[htb]
%     \centering
%     \includegraphics[width=0.5\textwidth]{your-figure.pdf}
%     \caption{Figure caption}
%     \label{fig:example}
% \end{figure}
% 
% See Figure \ref{fig:example} for details.

\end{document}