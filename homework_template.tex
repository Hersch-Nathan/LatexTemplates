% Homework Template - Basic Structure
% This is a minimal template to get you started with the homework.cls
% For a complete feature demonstration, see homeworktest.tex

\documentclass{homework}

\begin{document}

\hwheader{Course Name}{Assignment Number}{Due Date}{Your Name}

\problem{Problem Title}

Your solution goes here. You can use inline math like $x = 5$ or display math:
\begin{align}
    f(x) &= x^2 + 2x + 1 \\
    &= (x + 1)^2
\end{align}

\subproblem
Sub-problem solution here.

\subproblem
Another sub-problem solution here.

\problem{Code Example}

You can include code using the provided environments:

\begin{hwmatlab}
% MATLAB code example
x = linspace(0, 2*pi, 100);
y = sin(x);
plot(x, y);
\end{hwmatlab}

\begin{hwpython}
# Python code example
import numpy as np
x = np.linspace(0, 2*np.pi, 100)
y = np.sin(x)
\end{hwpython}

\begin{hwterminal}
$ python script.py
Output appears here
\end{hwterminal}

\problem{Including Graphics}

You can include figures and reference them:

% \begin{figure}[htb]
%     \centering
%     \includegraphics[width=0.5\textwidth]{your-figure.pdf}
%     \caption{Figure caption}
%     \label{fig:example}
% \end{figure}
% 
% See Figure \ref{fig:example} for details.

\problem{Block Diagrams}

Create centered block diagrams using the hwblocks environment:

\begin{hwblocks}
\bXInput{A} 
\bXBlocL{B}{$G(s)$}{A} 
\bXOutput{C}{B}
\bXLink{A}{B} 
\bXLink{B}{C}
\end{hwblocks}

\end{document}