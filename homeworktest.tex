%Homework Class Complete Feature Showcase
%This file demonstrates every feature available in homework.cls
%Author: Homework Class Template
%Date: September 1, 2025

\documentclass{homework}

\begin{document}

% Feature 1: Document Header - \hwheader{Course}{Assignment}{Date}{Name}
\hwheader{EE571}{Feature Demo}{2025-09-01}{Template Showcase}

% Feature 2: Basic Problem with Automatic Numbering
\problem{This demonstrates the basic problem command with automatic numbering.}

This is how you create a basic problem. The numbering happens automatically (Problem 1, Problem 2, etc.).

% Feature 3: Problem with Custom Numbering
\problem[A.1]{This demonstrates custom problem numbering.}

You can override the automatic numbering by providing a custom label in square brackets.

% Feature 4: Problem with Question Text
\problem{Find the derivative of $f(x) = 3x^2 + 2x - 5$ and evaluate it at $x = 2$.}

This shows how to include the actual question or description within the problem command.

Using the power rule:
\begin{hwmath}
f(x) \eq 3x^2 + 2x - 5 \\
f'(x) \eq 6x + 2 \\
f'(2) \eq 6(2) + 2 \eq 12 + 2 \eq 14
\end{hwmath}

% Feature 5: Sub-problems with Automatic Lettering
\problem{This problem demonstrates sub-problems with multiple parts.}

\subproblem{Find the transfer function of the system.}

Given the differential equation $\ddot{y} + 3\dot{y} + 2y = 5u(t)$:

\begin{hwmath}
s^2Y(s) + 3sY(s) + 2Y(s) \eq 5U(s) \\
G(s) \eq \frac{Y(s)}{U(s)} \eq \frac{5}{s^2 + 3s + 2}
\end{hwmath}

\subproblem{Factor the denominator and find the poles.}

\begin{hwmath}
s^2 + 3s + 2 \eq (s + 1)(s + 2) \\
\text{Poles: } s_1 \eq -1, \quad s_2 \eq -2
\end{hwmath}

\subproblem[iii]{Determine system stability (custom sub-problem numbering).}

Since both poles have negative real parts, the system is stable.

% Feature 6: hwmath Environment - Custom Math with Alignment
\problem{Demonstrate the hwmath environment for aligned equations.}

The hwmath environment provides automatic alignment on equals signs using the \texttt{\\textbackslash eq} shorthand:

\begin{hwmath}
\mathcal{L}\{f(t)\} \eq F(s) \\
\mathcal{L}\{\dot{f}(t)\} \eq sF(s) - f(0) \\
\mathcal{L}\{\ddot{f}(t)\} \eq s^2F(s) - sf(0) - \dot{f}(0)
\end{hwmath}

% Feature 7: hwmathnumbered Environment - Numbered Equations
\problem{Show numbered equations for referencing.}

For equations that need to be referenced later, use hwmathnumbered:

\begin{hwmathnumbered}
E \eq mc^2 \\
F \eq ma \\
P \eq VI
\end{hwmathnumbered}

These equations are numbered automatically and can be referenced in the text.

% Feature 8: Complex Mathematical Expressions
\problem{Demonstrate complex mathematical formatting capabilities.}

\subproblem{Matrix operations and state-space representation.}

\begin{hwmath}
\dot{\mathbf{x}} \eq \mathbf{A}\mathbf{x} + \mathbf{B}\mathbf{u} \\
\mathbf{y} \eq \mathbf{C}\mathbf{x} + \mathbf{D}\mathbf{u}
\end{hwmath}

Where the system matrices are:
\begin{hwmath}
\mathbf{A} \eq \begin{bmatrix}
0 & 1 & 0 \\
0 & 0 & 1 \\
-2 & -3 & -4
\end{bmatrix}, \quad
\mathbf{B} \eq \begin{bmatrix}
0 \\ 0 \\ 1
\end{bmatrix}, \quad
\mathbf{C} \eq \begin{bmatrix}
1 & 0 & 0
\end{bmatrix}
\end{hwmath}

\subproblem{Convolution integral and Laplace transforms.}

The convolution of two functions:
\begin{hwmath}
(f \ast g)(t) \eq \int_{-\infty}^{\infty} f(\tau)g(t-\tau) d\tau
\end{hwmath}

Laplace transform definition:
\begin{hwmath}
\mathcal{L}\{f(t)\} \eq \int_{0}^{\infty} f(t)e^{-st} dt
\end{hwmath}

% Feature 9: Note Command for Highlighting
\problem{Demonstrate the note command for highlighting important information.}

The note command displays text in bold, red, and large font:

\note[IMPORTANT: Check all calculations before submitting]

\note[TODO: Add numerical verification]

\note[FIX ME: Verify units are consistent]

\note  % This displays the default "NOTE"

This is useful for marking sections that need attention during review.

% Feature 10: MATLAB Code Environment
\problem{Demonstrate the hwmatlab environment for MATLAB code.}

\subproblem{MATLAB function with caption.}

\begin{hwmatlab}[caption=Second Order System Analysis]
function [y, t] = second_order_response(wn, zeta, K, tfinal)
    % Calculate step response of second order system
    % wn: natural frequency, zeta: damping ratio, K: gain
    
    s = tf('s');
    G = K * wn^2 / (s^2 + 2*zeta*wn*s + wn^2);
    
    [y, t] = step(G, tfinal);
    
    figure;
    plot(t, y, 'LineWidth', 2);
    xlabel('Time (s)');
    ylabel('Amplitude');
    title('Step Response');
    grid on;
end
\end{hwmatlab}

\subproblem{Simple MATLAB commands without caption.}

\begin{hwmatlab}
% Define system parameters
wn = 5;        % Natural frequency (rad/s)
zeta = 0.3;    % Damping ratio
K = 2;         % Gain

% Calculate and plot response
[y, t] = second_order_response(wn, zeta, K, 2);
\end{hwmatlab}

% Feature 11: Python Code Environment
\problem{Demonstrate the hwpython environment for Python code.}

\subproblem{Python implementation with caption.}

\begin{hwpython}[caption=Control Systems Analysis in Python]
import numpy as np
import matplotlib.pyplot as plt
import control as ct

def second_order_response(wn, zeta, K, tfinal):
    """
    Calculate step response of second order system
    Parameters:
        wn: natural frequency (rad/s)
        zeta: damping ratio
        K: gain
        tfinal: final time for simulation
    """
    
    # Create transfer function
    num = [K * wn**2]
    den = [1, 2*zeta*wn, wn**2]
    G = ct.tf(num, den)
    
    # Calculate step response
    t = np.linspace(0, tfinal, 1000)
    y, t = ct.step_response(G, t)
    
    # Plot results
    plt.figure(figsize=(10, 6))
    plt.plot(t, y, linewidth=2, label='Step Response')
    plt.xlabel('Time (s)')
    plt.ylabel('Amplitude')
    plt.title(f'Second Order System (wn={wn}, zeta={zeta}, K={K})')
    plt.grid(True, alpha=0.3)
    plt.legend()
    plt.show()
    
    return y, t
\end{hwpython}

\subproblem{Python data analysis without caption.}

\begin{hwpython}
# System parameters
wn = 5.0      # Natural frequency
zeta = 0.3    # Damping ratio  
K = 2.0       # Gain

# Performance metrics
overshoot = np.exp(-zeta*np.pi/np.sqrt(1-zeta**2)) * 100
settling_time = 4 / (zeta * wn)
peak_time = np.pi / (wn * np.sqrt(1 - zeta**2))

print(f"Overshoot: {overshoot:.1f}%")
print(f"Settling time: {settling_time:.2f} s")
print(f"Peak time: {peak_time:.2f} s")
\end{hwpython}

% Feature 12: Terminal Output Environment
\problem{Demonstrate the hwterminal environment for terminal output.}

The hwterminal environment displays terminal commands and output with appropriate styling:

\begin{hwterminal}[caption=MATLAB Simulation Output]
>> wn = 5; zeta = 0.3; K = 1;
>> s = tf('s');
>> G = K*wn^2/(s^2 + 2*zeta*wn*s + wn^2);
>> step(G);
>> stepinfo(G)

ans = 
       RiseTime: 0.2927
    SettlingTime: 2.6667
     SettlingMin: 0.9000
     SettlingMax: 1.3625
       Overshoot: 36.2467
      Undershoot: 0
            Peak: 1.3625
        PeakTime: 0.6283
\end{hwterminal}

\begin{hwterminal}[caption=Python Package Installation]
$ pip install control matplotlib numpy
Collecting control
  Downloading control-0.9.4-py3-none-any.whl (719 kB)
     ======================================== 719.0/719.0 kB 5.2 MB/s
Collecting matplotlib
  Downloading matplotlib-3.7.2-cp311-cp311-macosx_10_12_x86_64.whl (7.8 MB)
     ======================================== 7.8/7.8 MB 8.4 MB/s
Successfully installed control-0.9.4 matplotlib-3.7.2 numpy-1.24.3
\end{hwterminal}

% Feature 13: Graphics Inclusion Command
\problem{Demonstrate the hwgraphic command for including images.}

The \texttt{hwgraphic} command provides a convenient wrapper for including graphics with automatic centering:

\subproblem{Command syntax and features.}

\textbf{Basic syntax:}
\begin{center}
\texttt{\textbackslash hwgraphic\{path\}[optional title][optional scale]}
\end{center}

\textbf{Key features:}
\begin{itemize}
\item Automatic centering of images
\item Optional title with bold formatting
\item Default sizing to 80\% of text width
\item Custom scaling support with scale parameter
\item Works with PNG, JPG, PDF, and EPS formats
\end{itemize}

\textbf{Usage examples:}
\begin{itemize}
\item \texttt{\textbackslash hwgraphic\{figure.png\}} - Basic inclusion with default size
\item \texttt{\textbackslash hwgraphic\{plot.pdf\}[System Response]} - With title
\item \texttt{\textbackslash hwgraphic\{diagram.jpg\}[Circuit Diagram][0.6]} - With title and 60\% scale
\end{itemize}

\note[NOTE: Actual image files must exist for the hwgraphic command to work in practice]

% Feature 14: Advanced Mathematical Concepts
\problem{Showcase advanced mathematical typesetting capabilities.}

\subproblem{Partial derivatives and multivariable calculus.}

\begin{hwmath}
\frac{\partial^2 u}{\partial x^2} + \frac{\partial^2 u}{\partial y^2} \eq 0 \quad \text{(Laplace's equation)} \\
\nabla^2 u \eq \frac{\partial^2 u}{\partial x^2} + \frac{\partial^2 u}{\partial y^2} + \frac{\partial^2 u}{\partial z^2} \eq 0
\end{hwmath}

\subproblem{Complex analysis and frequency domain.}

\begin{hwmath}
H(j\omega) \eq \frac{K}{j\omega + a} \\
|H(j\omega)| \eq \frac{K}{\sqrt{\omega^2 + a^2}} \\
\angle H(j\omega) \eq -\arctan\left(\frac{\omega}{a}\right)
\end{hwmath}

% Feature 15: Piecewise Functions and Special Cases
\problem{Demonstrate piecewise functions and special mathematical constructs.}

\subproblem{Unit step and impulse functions.}

\begin{hwmath}
u(t) \eq \begin{cases}
1 & t \geq 0 \\
0 & t < 0
\end{cases} \\
\delta(t) \eq \begin{cases}
\infty & t = 0 \\
0 & t \neq 0
\end{cases}, \quad \int_{-\infty}^{\infty} \delta(t) dt \eq 1
\end{hwmath}

\subproblem{System response types based on damping.}

\begin{hwmath}
\text{Response type} \eq \begin{cases}
\text{Overdamped} & \zeta > 1 \\
\text{Critically damped} & \zeta = 1 \\
\text{Underdamped} & 0 < \zeta < 1 \\
\text{Undamped} & \zeta = 0
\end{cases}
\end{hwmath}

% Feature 16: Integration with Standard LaTeX Packages
\problem{Show integration with standard LaTeX environments and packages.}

The homework class works seamlessly with standard LaTeX features:

\begin{itemize}
\item Standard lists and enumerations
\item Mathematical symbols: $\alpha, \beta, \gamma, \Delta, \Omega$
\item Greek letters and special symbols: $\infty, \partial, \nabla, \sum, \int$
\item Text formatting: \textbf{bold}, \textit{italic}, \texttt{monospace}
\end{itemize}

\begin{enumerate}
\item First numbered item
\item Second numbered item with math: $E = mc^2$
\item Third item with inline code: \texttt{plot(x, y)}
\end{enumerate}

% Feature 17: Final Comprehensive Example
\problem{Comprehensive example combining all features.}

\note[FINAL EXAMPLE: Combines multiple features]

\subproblem{Design a control system.}

Given plant: $G_p(s) = \frac{10}{s(s+2)}$

Design a PID controller: $G_c(s) = K_p + \frac{K_i}{s} + K_d s$

\begin{hwmathnumbered}
G_{ol}(s) \eq G_c(s)G_p(s) \\
G_{cl}(s) \eq \frac{G_{ol}(s)}{1 + G_{ol}(s)}
\end{hwmathnumbered}

\subproblem{MATLAB implementation.}

\begin{hwmatlab}[caption=PID Controller Design and Analysis]
% Plant transfer function
s = tf('s');
Gp = 10 / (s * (s + 2));

% PID controller parameters
Kp = 5; Ki = 2; Kd = 0.5;
Gc = pid(Kp, Ki, Kd);

% Closed-loop system
Gcl = feedback(Gc * Gp, 1);

% Analysis
step(Gcl);
title('Closed-Loop Step Response');
grid on;

% Display controller
fprintf('PID Controller: Kp=%.1f, Ki=%.1f, Kd=%.1f\n', Kp, Ki, Kd);
\end{hwmatlab}

\subproblem[*]{Bonus: Python verification.}

\begin{hwpython}[caption=Python Control Systems Verification]
import control as ct
import numpy as np
import matplotlib.pyplot as plt

# Plant transfer function
Gp = ct.tf([10], [1, 2, 0])

# PID controller
Kp, Ki, Kd = 5, 2, 0.5
Gc = ct.tf([Kd, Kp, Ki], [1, 0])

# Closed-loop system  
Gcl = ct.feedback(Gc * Gp, 1)

# Step response
t = np.linspace(0, 5, 1000)
y, t = ct.step_response(Gcl, t)

plt.figure(figsize=(10, 6))
plt.plot(t, y, 'b-', linewidth=2)
plt.xlabel('Time (s)')
plt.ylabel('Output')
plt.title('PID Controlled System Step Response')
plt.grid(True, alpha=0.3)
plt.show()

print(f"Controller: PID({Kp}, {Ki}, {Kd})")
\end{hwpython}

This example demonstrates the complete integration of mathematical typesetting, code environments, problem structuring, and highlighting features available in the homework class.

% Appendix demonstration
\hwappendix
This is the first appendix. It contains additional derivations and supplementary material.

\begin{hwmath}
\text{Extended derivation of the quadratic formula:} \\
ax^2 + bx + c = 0 \\
ax^2 + bx = -c \\
x^2 + \frac{b}{a}x = -\frac{c}{a} \\
x^2 + \frac{b}{a}x + \left(\frac{b}{2a}\right)^2 = -\frac{c}{a} + \left(\frac{b}{2a}\right)^2 \\
\left(x + \frac{b}{2a}\right)^2 = \frac{b^2 - 4ac}{4a^2} \\
x + \frac{b}{2a} = \pm\frac{\sqrt{b^2 - 4ac}}{2a} \\
x = \frac{-b \pm \sqrt{b^2 - 4ac}}{2a}
\end{hwmath}

\hwappendix[Reference Tables]
This appendix contains useful reference tables and formulas.

\begin{table}[h]
\centering
\begin{tabular}{|c|c|c|}
\hline
\textbf{Transform} & \textbf{Time Domain} & \textbf{Frequency Domain} \\
\hline
Unit Step & $u(t)$ & $\frac{1}{s}$ \\
Ramp & $t \cdot u(t)$ & $\frac{1}{s^2}$ \\
Exponential & $e^{-at} \cdot u(t)$ & $\frac{1}{s+a}$ \\
Sine & $\sin(\omega t) \cdot u(t)$ & $\frac{\omega}{s^2 + \omega^2}$ \\
\hline
\end{tabular}
\caption{Common Laplace Transform Pairs}
\end{table}

\hwappendix
This is the third appendix, showing automatic lettering continues (Appendix C).

\note[Appendices are useful for including detailed calculations, reference materials, or code listings that would interrupt the flow of the main document.]

\end{document}
